\englishabstracttitle{Research on Fault-tolerant Properties of Augmented Cubes}
\begin{englishabstract}

With a series of fruitful achievements in supercomputers in China, high performance computing has attracted more and more attention. Due to the huge data processing and numerical computing capacity, high performance computing is widely used in national defense construction and national economy, which is one of the important symbols to measure a country's comprehensive national strength. As an important part of high performance computing, the performance of interconnection network is very important for high performance computing.

There are many factors that determine the performance of interconnection network. Fault tolerance and reliability are important reference standards to measure the performance of an interconnection network. An interconnection network with good fault tolerance will not affect the performance of the whole network even if some servers in the network fail. Connectivity is one of the important indexes to measure a interconnection network. But the traditional connectivity often assumes that when a vertex in the network fails, it has no effect on the surrounding vertices. However, in the actual running environment, when a vertex fails, the probability of the failure of its surrounding vertices is also greatly increased. In order to make up for this defect, structure connectivity and substructure connectivity consider the fault-tolerant ability of interconnection network from the perspective of fault structure rather than fault vertex. On the other hand, when the vertices in the network fail, how to diagnose them accurately and replace them with fault-free vertices to ensure the normal operation of the interconnection network is also an important aspect of the interconnection network. The maximum number of fault vertices that an interconnection network can determine is called the diagnosability. Diagnosability is also a key index to measure the fault tolerance performance of interconnection network.


The augmented cube is an important variant of the famous interconnection network hypercube. The augmented cube not only retains many of the great properties of the hypercube, but also has advantages that the hypercube and its other variants do not have. For example, the connectivity of the augmented cube is $2n-1$, almost twice that of the hypercube. This also means that the fault tolerance of the augmented cube is in some ways much better than that of the hypercube.

In this paper, we mainly study the properties of fault tolerance of augmented cube, including structure connectivity, substructure connectivity and extra diagnosability. At the same time, the fault-tolerant routing algorithm and diagnosis algorithm are proposed. Specific achievements are as follows:

(1) In this paper, the structure connectivity and substructure connectivity of the augmented cube are studied from the point of view of the fault structure rather than the fault vertex. At the same time, a fault-tolerant routing algorithm is proposed under $K_{1,6}$-substructure connectivity. And the validity and accuracy of the algorithm are verified by simulation experiments.

(2) Under the definition of extra diagnosability, we study the  $2$-extra diagnosability of the augmented cube and give detailed proof. At the same time, we proposed an additional diagnosis algorithm based on $2$-extra diagnosability. This algorithm can help us locate the fault vertex correctly to ensure the normal operation of the network.





\englishkeywords{High performance computing, Augmented cube, Structure connectivity, Substructure connectivity, Extra connectivity, Extra diagnosability, Fault-tolerant routing algorithm, Diagnosis algorithm}

\end{englishabstract}


\ensoochowauthor{Shuangxiang Kan}

\ensoochowtutor{Janxi Fan}

